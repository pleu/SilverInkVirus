% request for removal from telemarketing lists

\documentclass[12pt]{letter}

\newcommand{\cX}{{\mathcal X}}

\usepackage{gensymb}
\usepackage[version=3]{mhchem} % Formula subscripts using \ce{}
%\usepackage{hyperref}
\usepackage[normalem]{ulem}
\usepackage{textcomp}
\usepackage{color}
\usepackage{xcolor}
%\usepackage{pdfcomment}
\usepackage{amsmath}
\usepackage{amssymb}
\usepackage{pdfcomment}

\usepackage{paralist}

\usepackage{setspace}
\usepackage{hyperref}
%\usepackage{color}
\usepackage{textcomp}
\usepackage{amsmath}
\usepackage{ulem}
\usepackage{graphicx}


%\definecolor{blue}{rgb}{0,0,0.543}
%\definecolor{blue}{rgb}{0,0.08,0.45}
%\def\blue[rgb]{0 0 139}

\oddsidemargin=.2in
\evensidemargin=.2in
\textwidth=5.9in
\topmargin=-.5in
\textheight=9in

\usepackage{xargs}
\usepackage[colorinlistoftodos,prependcaption,textsize=tiny]{todonotes}


%\newcommand{\blue}[1]{\textcolor{blue}{#1}} %for displaying red texts
%\newcommand{\strikeout}[1]{\textcolor{blue}{\sout{#1}}} %for displaying red texts


\newcommandx{\commentPaul}[2][1=]{\todo[linecolor=red,backgroundcolor=red!25,bordercolor=red,#1]{#2}}

\defineavatar{Paul}{color=red}


%\newcommandx{\commentPaul}[2][1=]{\todo[linecolor=red,backgroundcolor=red!25,bordercolor=red,#1]{#2}}


\newcommand{\strikeout}[1]{\textcolor{blue}{\sout{#1}}} %for displaying red texts

\newcommand{\red}[1]{\textcolor{red}{#1}} %for displaying red texts
\newcommand{\blue}[1]{\textcolor{blue}{#1}} %for displaying red texts
\newcommand{\Hmax}{\hat{H}}

\onehalfspacing
%\name{Sajad Haghanifar}
\name{Paul W Leu}

\date{\today}

\begin{document}

\begin{letter}{}

\opening{Dear Dr.~Editor:}


The article in part has previously been submitted to ACS Applied Materials \& Interfaces. We thank the two reviewers for their evaluation which has helped us revise the manuscript and better emphasize its novelty compared to previous research.  Both reviewers mention previous research that has been performed on  silver nanoparticles in textiles.  It should be emphasized that our research focuses on reactive silver inks for textiles and their improved antiviral performance and wash durability compared to silver nanoparticles.
Our work focuses on antiviral activity after washing, which has not been previously reported for silver nanoparticles.  
Additionally, in our submitted paper, we test the fabric treatments after extended bleach washing with a harsh ultrasonication washing procedure under high concentration bleach (10\%) which has not been previously reported. 

%to allow our work to be of interest 
%to the readership of ACS Applied Nano Materials.

% We thank the three reviewers for their evaluation of our manuscript and comments.  
%We thank the reviewers' statements that ``this review article is comprehensive to introduce the various sources of bio-inspiration, the critical performance requirements and the strategies that have been carried out to tackle the challenges" and we have provided ``an elaborate and comprehensive arrangement of various transparent substrates and barrier layers for optoelectronics." 

We have made improvements to our paper based on the reviewers' comments and suggestions.
Below, we summarize the changes that have been made and respond to the reviewers' comments.
Considering the importance of the topic of this
article on 
%for application of 
bleach washable, anti-viral fabric coatings,
we resubmit this manuscript for your consideration of publication in ACS Applied Bio Materials.
Thank you for your consideration.

\closing{My best regards,}

%Paul Leu



\end{letter}



%\newpage

%\textbf{Editor}


%\begin{enumerate}

%\item 
%Please add a 
%scheme near the 
%beginning (probably 
%defined as Scheme 1) 
%to illustrate the 
%system 
%(structure/chemical 
%composition) and the 
%synthesis/fabrication 
%from a chemistry 
%perspective.

%blue{A schematic showing the synthesis and structure of the functionalized nano graphene oxide has been added to Figure 1.}

%\item 
%Please remove the box 
%around the TOC graphic and 
%increase its clarity/resolution.

%\blue{The box has been removed and the resolution of the graphic has been improved by using a .tif file.}

%\item In Figures 1c, 5c, S5, and 2c please use a.u. for the units and not use A.u.

%\blue{This has been corrected.}

%item The contact angle data in Figures 3a and 5e are reported with too much precision (i.e., too many significant figures). Contact angles should be reported as xx degrees not xx.y degrees. Please check carefully and revise where needed. 

%\blue{This has been corrected.}

%\item Please check Figure S4, some of it is missing.

%\blue{This has been corrected.}

%\item Please reformat the Supporting Information so that each figure appears on the same page as its figure caption.

%\blue{This has been corrected.}

%\end{enumerate}

\newpage

\textbf{Reviewer: 1}

\textbf{Recommendation:} Not appropriate for ACS Applied Materials and Interfaces.

\textbf{Comments:}
I can not see ant novelties in this paper.
Please read the following paper:
Colloids and Surfaces A: Physicochemical and Engineering 
Aspects
Volume 345, Issues 1–3, 5 August 2009, Pages 202-210
Colloids and Surfaces A: Physicochemical and Engineering 
Aspects
A new method to stabilize nanoparticles on textile surfaces
Roya Dastjerdi, Majid Montazer, Shadi Shahsavan

\url{https://doi.org/10.1016/j.colsurfa.2009.05.007}

\blue{We thank the reviewer for pointing out this paper to us.  While this paper by Dastjerdi \textit{et.~al} does have some relevance to our submitted paper, it should be emphasized that the Dastjerdi \textit{et.~al} paper focuses on silver nanoparticles and stabilizing them using polysiloxane.  In contrast, our paper focuses on reactive silver inks and their improved performance compared to silver nanoparticles.  
This Dastjerdi \textit{et.~al} paper does NOT actually test the washing durability of the treatment, but simply claims that the washing durability is improved. Additionally, their previous work investigates antibacterial activity.\\ \\
It should be noted that there is a lot of research in the literature on silver nanoparticles and improving their wash durability.  We have attempted to capture this previous research through the following sentences at the end of the second paragraph, ``Significant research has been dedicated to improving the adhesion forces of silver nanoparticles with textile fibers such as the use of polymer coatings,$^{21}$ polymer binders,$^{22}$ binders,$^{23}$ silanization,$^{24}$ and nanoparticle/polymer composites.$^{25}$ Nevertheless,  nanoparticle-based methods are inherently limited by the non-uniformity of the particle deposition and difficulties in controling silver release.$^{26,27}$"\\ \\
Our work utilizes reactive silver ink to investigate its antiviral activity and compares reactive silver ink to silver nanoparticles. Additionally, our work tests the antiviral activity of the fabric treatments after extended bleach washing with a harsh ultrasonication washing procedure.  
The novelty of this work has been highlighted by changing the title and improving the third paragraph of the paper in addition to the aforementioned sentences at the end of the second paragraph. The harsh washing procedure which includes both ultrasonic agitation and high concentration bleach (10\%) used in this work is important for healthcare settings and has not been previously tested. The third paragraph now reads, 
``There has been no previous work on the antiviral properties of silver fabric treatmentsafter disinfection with bleach (sodium hypochlorite) concentrations as high as 10\% and ultrasonic cleaning.  Bleach (sodium hypochlorite) is a strong oxidant that is chemically reactive with organic molecules.$^{28,29}$ 
The Centers for Disease Control and Prevention (CDC) recommends cleaning and disinfecting surfaces with a bleach washing concentration up to 5,000 ppm (1:10 dilution of household bleach).$^{30}$ 
Reusable textiles in healthcare settings may similarly be decontaminated by washing in bleach solutions as high as 10\%;$^{30,31}$ therefore, it is essential to study the bleach wash durability of antimicrobial/antiviral techniques for medical fabric applications.  
This article utilizes the highest bleach concentration recommended by the CDC in the bleach washing procedure to ensure the potential for reusability in medical settings.  F
urthermore, ultrasonic cleaning induces cavitation bubbles to produce high mechanical forces on the fabric$^{32}$ and mechanical stress has been shown to play a dominantrole in the removal of silver from fabrics.$^{26}$ 
This paper evaluates fabrics cleaned by ultrasonic cleaning (under ASTM G131-96 standards), which has been shown to achieve better 
detergency.$^{33}$"
}

\blue{An improved explanation of the silver release through PDMS, where the mentioned previous work is lacking, has been added on page 13, lines 4-17. }

%\commentPaul{Think all page numbers and paragraph numbers need to be checked and changed as I think they have changed after modifying figures.}

% to introduce the various sources of bio-inspiration,the critical performance requirements and the strategies that have been carried out to tackle the challenges" and the content we provide is “solid and well drafted". We have made improvement based on reviewer's comments:}
% \blue{A more detailed review of previous durability methods has been added on page 3, lines 6-10. The washing durability of anti-viral silver fabric treatments has not been previously investigated. }
%}

%\begin{enumerate}


%\end{enumerate}

\newpage
\textbf{Reviewer: 2}

\textbf{Recommendation:} Not appropriate for ACS Applied Materials and Interfaces.

\textbf{Comments:} The authors reported the preparation of an anti-viral and water-repellent PET textile by depositing silver particles on the fibers of the textile and subsequent coverage of PDMS. The structure and properties of the sample were characterized, and the sample showed good washing durability. However, the silver particles for viral and bacterium and the PDMS for low-surface-energy superhydrophobic materials have been reported in many previous papers. Therefore, the work lacks obvious novelty. In addition, the writing equality of this manuscript is not good. Therefore, I can not recommend it for publication in ACS Applied Materials and Interfaces.

\blue{We thank the reviewer for his or her constructive comments. While the antiviral and antibacterial activity of silver particles have been reported in many previous papers, there has not been previous research on reactive silver inks and their antiviral properties.  It should be emphasized that our work focuses on reactive silver inks instead of silver nanoparticles.\\ \\    
Our work focuses on antiviral activity after washing, which has not been previously reported, even for silver nanoparticles.  
Additionally, this work tests the fabric treatments after extended bleach washing with a harsh ultrasonication washing procedure 
under high concentration bleach (10\%) which has not been previously reported.
%which has not been reported. % and evaluation of our article to be published in this prestigious journal. 
We have made improvements based on the reviewer's comments to highlight the novelty, improve the literature references, and enhance the writing quality. }


\blue{A more detailed review of previous durability methods has been added.
%on page 3, lines 6-10. 
The washing durability of anti-viral silver treatments has not been previously investigated. The novelty of this work has been highlighted by changing the title and adding the discussion on page 3, lines 11-24. The harsh washing procedure used in this work is important for healthcare settings and has not been previously tested.
Please see comments to Reviewer 1 for the new text that has been added to the paper.}

% \blue{An improved explanation of the silver release through PDMS, where the mentioned previous work is lacking, has been added on page 13, lines 4-17. }



%\end{enumerate}



\blue{In addition to these changes, we have also made minor changes throughout the paper in order to improve its writing quality and highlight the importance of this work.}



\newpage


\textbf{Submitting Authors:}\\
%Sajad Haghanifar\\
%Benedum Hall, Room SB10 \\
%Pittsburgh, PA 15261\\
%Email:  \href{mailto:sah175@pitt.edu}{sah175@pitt.edu}\\
%Phone: (412) 499-0768
Anthony J. Galante\\
Benedum Hall, Room SB10 \\
Pittsburgh, PA 15261\\
Email:  \href{mailto:ajg109@pitt.edu}{ajg109@pitt.edu}\\
Phone: (919) 757-4183\\

Brady C. Pilsbury \\
University of Pittsburgh \\
Department of Industrial Engineering \\
Benedum Hall, Room SB10 \\
Pittsburgh, PA 15261\\

Kathleen A. Yates\\
 203 Lothrop Street, Floor 10\\
Pittsburgh, PA 15213, USA\\
Email:  \href{mailto:kayates@pitt.edu}{kayates@pitt.edu}\\

Melbs LeMieux \\
Electroninks Inc \\ 
7901 East Riverside Drive, Bldg 1,Unit 150\\
Austin TX  78744 \\

Eric G. Romanowski\\
%\altaffiliation{A shared footnote}
 203 Lothrop Street, Floor 10\\
Pittsburgh, PA 15213, USA\\
Email:  \href{mailto:egr1@pitt.edu}{egr1@pitt.edu}\\
%\phone{+123 (0)123 4445556}
%\fax{+123 (0)123 4445557}

Robert M. Q. Shanks\\
 203 Lothrop Street, Floor 10\\
Pittsburgh, PA 15213, USA\\
Email:  \href{mailto:shanksrm@pitt.edu}{shanksrm@pitt.edu}\\

Paul W. Leu\\
Benedum Hall, Room 1035 \\
Pittsburgh, PA 15261\\
Email:  \href{mailto:pleu@pitt.edu}{pleu@pitt.edu}\\
Phone: (412) 624-9834

\textbf{Corresponding Author:}\\
Paul W. Leu\\
Benedum Hall Room 1035 \\
Pittsburgh, PA, 15261\\
Email:  \href{mailto:pleu@pitt.edu}{pleu@pitt.edu}\\
Phone: (412) 624-9834





%\bibliographystyle{unsrt}
%\bibliography{AllRefs,LAMPPapers}


\end{document} 
% request for removal from telemarketing lists

\documentclass[12pt]{letter}

\newcommand{\cX}{{\mathcal X}}

\usepackage{gensymb}
\usepackage[version=3]{mhchem} % Formula subscripts using \ce{}
%\usepackage{hyperref}
\usepackage[normalem]{ulem}
\usepackage{textcomp}
\usepackage{color}
\usepackage{xcolor}
%\usepackage{pdfcomment}
\usepackage{amsmath}
\usepackage{amssymb}
\usepackage{pdfcomment}

\usepackage{paralist}

\usepackage{setspace}
\usepackage{hyperref}
%\usepackage{color}
\usepackage{textcomp}
\usepackage{amsmath}
\usepackage{ulem}
\usepackage{graphicx}


%\definecolor{blue}{rgb}{0,0,0.543}
%\definecolor{blue}{rgb}{0,0.08,0.45}
%\def\blue[rgb]{0 0 139}

\oddsidemargin=.2in
\evensidemargin=.2in
\textwidth=5.9in
\topmargin=-.5in
\textheight=9in

\usepackage{xargs}
\usepackage[colorinlistoftodos,prependcaption,textsize=tiny]{todonotes}


%\newcommand{\blue}[1]{\textcolor{blue}{#1}} %for displaying red texts
%\newcommand{\strikeout}[1]{\textcolor{blue}{\sout{#1}}} %for displaying red texts


\newcommandx{\commentPaul}[2][1=]{\todo[linecolor=red,backgroundcolor=red!25,bordercolor=red,#1]{#2}}

\defineavatar{Paul}{color=red}


%\newcommandx{\commentPaul}[2][1=]{\todo[linecolor=red,backgroundcolor=red!25,bordercolor=red,#1]{#2}}


\newcommand{\strikeout}[1]{\textcolor{blue}{\sout{#1}}} %for displaying red texts

\newcommand{\red}[1]{\textcolor{red}{#1}} %for displaying red texts
\newcommand{\blue}[1]{\textcolor{blue}{#1}} %for displaying red texts
\newcommand{\Hmax}{\hat{H}}

\onehalfspacing
%\name{Sajad Haghanifar}
\name{Paul W Leu}

\date{\today}

\begin{document}

\begin{letter}{}

\opening{Dear Dr.~Editor:}

We thank the five reviewers for their evaluation of our manuscript and generally favorable reviews.  
%which has helped us revise the manuscript.  %Both reviewers mention previous research that has been performed on  silver nanoparticles in textiles.  It should be emphasized that our research focuses on reactive silver inks for textiles and their improved antiviral performance and wash durability compared to silver nanoparticles.
%Our work focuses on antiviral activity after washing, which has not been previously reported for silver nanoparticles and ink.  
%Additionally, in our submitted paper, we test the fabric treatments after extended bleach washing with a harsh ultrasonication washing procedure under high concentration bleach (10\%) which has not been previously reported. 
%to allow our work to be of interest 
%to the readership of ACS Applied Nano Materials.
% We thank the three reviewers for their evaluation of our manuscript and comments.  
%We thank the reviewers' statements that ``this review article is comprehensive to introduce the various sources of bio-inspiration, the critical performance requirements and the strategies that have been carried out to tackle the challenges" and we have provided ``an elaborate and comprehensive arrangement of various transparent substrates and barrier layers for optoelectronics." 
We have made improvements to our paper based on the reviewers' comments and suggestions.
Below, we summarize the changes that have been made and respond to the reviewers' comments.
Considering the importance of the topic of this
article on 
%for application of 
bleach washable, anti-viral fabric coatings,
we resubmit this manuscript for your consideration of publication in PLOS One.
Thank you for your consideration.

\closing{My best regards,}

%Paul Leu



\end{letter}



%\newpage

%\textbf{Editor}


%\begin{enumerate}

%\item 
%Please add a 
%scheme near the 
%beginning (probably 
%defined as Scheme 1) 
%to illustrate the 
%system 
%(structure/chemical 
%composition) and the 
%synthesis/fabrication 
%from a chemistry 
%perspective.

%blue{A schematic showing the synthesis and structure of the functionalized nano graphene oxide has been added to Figure 1.}

%\item 
%Please remove the box 
%around the TOC graphic and 
%increase its clarity/resolution.

%\blue{The box has been removed and the resolution of the graphic has been improved by using a .tif file.}

%\item In Figures 1c, 5c, S5, and 2c please use a.u. for the units and not use A.u.

%\blue{This has been corrected.}

%item The contact angle data in Figures 3a and 5e are reported with too much precision (i.e., too many significant figures). Contact angles should be reported as xx degrees not xx.y degrees. Please check carefully and revise where needed. 

%\blue{This has been corrected.}

%\item Please check Figure S4, some of it is missing.

%\blue{This has been corrected.}

%\item Please reformat the Supporting Information so that each figure appears on the same page as its figure caption.

%\blue{This has been corrected.}

%\end{enumerate}

\newpage
\textbf{Journal Requirements:}

1. Please ensure that your manuscript meets PLOS ONE's style requirements, including those for file naming.

\blue{The PLOS One requirements are met to the best of our knowledge. }

2. We note that you have stated that you will provide repository information for your data at acceptance. 
Should your manuscript be accepted for publication, we will hold it until you provide the relevant accession numbers or DOIs necessary to access your data. 
If you wish to make changes to your Data Availability statement, please describe these changes in your cover letter and we will update your Data Availability statement to reflect the information you provide.

\blue{We understand.}

3. Please amend the manuscript submission data (via Edit Submission) to include author Melbs LeMieux.

\blue{The submission portal will not allow us to  add Melbs as an author because his affiliation is not recognized. Can his affiliation be added to the system or added manually?}

\textbf{Additional Editor Comments:}

This manuscript reports the use of reactive inks containing silver nanoparticles for improving antiviral, hydrophobic properties and durability (against harsh washing) of medical textiles. It is an interesting study for wider readership of scientific community working in medical textiles, nanotechnology and polymers. However, it requires revision before it can be considered further.

\blue{We thank the editor for recognizing the importance of our work in the fields of medical textiles, nanotechnology, and polymers.  We have revised the manuscript based on the editor and reviewers' suggestions.}

In addition reviewers comments, authors are requested to add the graphical abstract.

\blue{A graphical abstract has been added.}

Please update with relevant and recent works; e.g.
https://doi.org/10.1007/s10570-021-04257-z
https://doi.org/10.1080/00405000.2021.1996730

\blue{Citations of these recent works have been added to the text.}

\textbf{Reviewer: 1}

\textbf{Recommendation:} This paper investigates the use of silver nanoparticle and inks incorporated into knitted medical textiles that show durable anti-viral activity, even after extended washing in bleach. 
The paper is interesting, generally well-written and organised. 
However, several issues need to be addressed before it can be considered for publication. Thus, I would recommend major changes to the manuscript before resubmission and reconsideration.

\textbf{Comments:}

INTRODUCTION:
- Certain clarifications should be offered in the claims made. For example, not all bleaches are oxidising agents – the claim is written a little too broadly in the introduction. Perhaps rephrase to specify that most commonly encountered bleaches tend to be oxidising.

\blue{This statement has been rephrased.}

METHODOLOGY:
- Is more information available about the knitted substrate? For example, warp or weft knit? What kind of knit structure? Double or single?

\blue{The structure is single knit and this description has been added to the text.}

- Acquisition parameters for various instruments are missing or lacking in detail.
o XRD: Step size? Scan rate? Excitation source?
o FTIR: Number of scans? Step size?

\blue{The acquisition parameters of XRD and FTIR have been added to the experimental section.}

RESULTS:
- I think the data in Figure 5 needs further explanation. For example, the PET-NP/P after 100 mins of bleaching seemingly performs better than the unwashed equivalent. This is further made more difficult to understand based on the data from Figure 2 which indicates that near 100\% of the active species that supposedly confer anti-viral activity, are lost after 100 mins of bleach washing. It is incumbent on the authors to provide a full explanation and reasoning for the effects and trends observed.

\blue{ The observed values for the PET-NP/P unwashed compared to after 100 mins washing are not statistically different. Figure 2 corresponds to PET-NP and PET-Ink samples, not PET-NP/P samples. Further discussion on Figure 5 has been added for clarity.}

FIGURES:
- XRD spectra should have the correct x-axis labelleing. ‘Angle’ is insufficient.
- For figure 4, the SEM images are useful, but I think an inset with much higher magnification is required for the reader to ascertain if any changes to the nanoparticle loading etc., has occurred. It is not possible to do so at the 10 micron level.
- Perhaps a minor point, but the authors seem to have multiple different axis labelling styles; sometimes using brackets and other times using a ‘/’ between the label and units. A single labelling style is recommended for all figures.

\blue{The above mentioned corrections have been made. The authors choose to keep the axis labelling style. }


REFERENCES:
- 33 needs correcting
%\begin{enumerate}
%\end{enumerate}

\blue{Reference 33 does not have any more details. }

\newpage
\textbf{Reviewer: 2}

\textbf{Recommendation:} A reactive silver ink coating was developed to provide a bleach wash durable, repellent antiviral PET fabrics. Silver nanoparticle or reactive silver ink showed antiviral property that is very important for medical textiles. However, there are significant flaw in the present article.

\textbf{Comments:} 
1.Uniform distribution of silver on the PET fabrics is crucial to the good antiviral property. How did the authors adjust silver distribution via drop casting method? How about the silver content on the coating? Cytotoxicity is needed to characterize the coated PET fabrics.

\blue{PET samples were dampened with alcohol to improve the spreading of the silver coatings during drop casting. The silver content of the coating has been added in the Materials and Methods section. ``Future work should analyze the antiviral mechanism, as well as the cytotoxicity of reactive silver ink treated fabrics before
commercial use" has been added on line 253 in the Conclusion. }

2.The authors need to characterize PDMS coatings, for example, the thickness of PDMS coating? How did reactive silver ink and silver nanoparticle influence the final morphology of PDMS coatings?

\blue{A description of the final morphology of samples after PDMS coatings have been added in page 4, paragraph 4 line 150 ,``The PDMS 150
coating appears roughened due to the underlying morphology of the silver nanoparticles 151
or silver ink coated fabrics." }

3.Silver is the antiviral agent to defend against virus, so why PDMS treatment on the reactive silver ink coating had better antiviral property than silver ink coating without PDMS treatment for unwashed PET fabrics?

\blue{The PDMS coating adds a repellency property to the fabric, so there is simultaneous virus deactivation and repellency. ``The PDMS coated silver 
fabrics may release less silver than without PDMS; however, a greater reduction of 
infectious virus is observed with PDMS from a combination of virus inactivation and 
liquid repellency" has been added to page 6, bottom of paragraph 2, line 224-226.}

\newpage
\textbf{Reviewer: 3}

\textbf{Recommendation:} This is a well-written article covering a nice piece of timely research, at a moment in history when there is considerable interest in antiviral textiles. The quality of the science is high and the conclusions from the article are generally very robust. However, I have a small number of concerns that should be addressed through minor revision of the MS, mainly in highlighting the limitations of the study. i.e. One might be forgiven from interpreting the work and resulting technology as being at a late stage of technical readiness so the following limitations should be transparently disclosed in the article:

\textbf{Comments:} 
1. The efficacy of the reactive silver approach is clear, but are there any significant unknown risks such as toxicity or has the oxidation to elemental silver been proven to be 100\% reliable?

\blue{The ink leaves a pure silver film, so toxicity should not be different than elemental silver. The curing temperature is low and at this temperature the silver is not oxidised during curing. The XRD characterization in Figure S2 shows the silver is not oxidized on the fabric after curing. }

2. The washfastness involved single exposure experiments in one treatment regime (bleach in a sonicator) when there is little evidence provided that this is industrially relevant. Such a process provides little mechanical action, abrasion or shear forces, and certainly doesn't replicate autoclave treatments used in some areas for medical textiles. Claims are made that the technology is suitable for multiple washes and uses, but no evidence is provided to support this. Moreover, the control fabric is also lab produced - a robust demonstration of improved washfastness might have included a commercially purchased antiviral textile? In summary, far more studies would be needed before any conclusion was reached that this technology was approaching commercial viability.

\blue{Ultrasonication can produce harsh mechanical action and scrubbing via bubbles and can be industrially relevant.$^34$ ``Future work should analyze the antiviral mechanism, as well as the cytotoxicity of reactive silver ink treated fabrics before
commercial use" has been added on line 253 in the Conclusion.  }

3. Breadth - medical textiles are produced in more fabric types than PET knit yet no data is presented showing that the technology is applicable on other fabric types. We should therefore acknowledge the narrowness of the study.

\blue{Future work may consider other textiles and this limitation is acknowledged.  }

4. Other properties - technical textiles don't just need to exhibit their desired primary benefit (in this case antiviral); does the treatment compromise any other aspects of textile properties that could be relevant to its use such as flexibility and comfort in wear?

\blue{Future work may consider other textile properties and this limitation is acknowledged.  }

Finally, the descriptions of materials in the Materials and Methods are too vague and this prevents the work from being easily duplicated. Trademarks such as Antocon(R) are not acknowledged which is a product brand name not a supplier. That range of wipes contains more than one PET knit wipe so the specific one used should be described. The specific bleach is not given, only that it was purchased from Giant Eagle - was it Clorox(R) or their own brand? Do they really only sell one? Could product codes be provided for other purchases from Sigma-Aldrich unless they really only sell one PBS, FBS and PDMS product?

\blue{The supplier has been added, as well as a better description of . The specific bleach is Giant Eagle brand and (Giant Eagle brand) has been added to the text. They only sell one. A better description of the other materials that were used have been changed in the Materials and Methods section. Sigma only sells one PBS and FBs. PDMS (Sylgard 184) has been added to the text. }

\newpage
\textbf{Reviewer: 4}

\textbf{Recommendation:} 
The manuscript entitled "Reactive silver inks for antiviral, repellent medical textiles with ultrasonic bleach washing durability compared to silver nanoparticles" adequately studies and discusses the antiviral activity of silver coated PET fabrics with different silver sources in terms of wash durability. However, the manuscript seems to be unclear, making the technical and presentation aspects appear weak. The manuscript can make a good case for PLOS ONE readers with some suggested improvements. Below are issues of significant importance that should be addressed in the proposed manuscript.

\textbf{Comments:} 

*Fig 1 highlights the characterization .... with silver nanoparticles and inks. The authors should use terminology such "reactive silver inks" or "silver inks” instead of inks through their manuscript.

\blue{The terminology has been changed throughout the text. }

*An equal amount of silver by weight (0.2 mg) was added to each fabric. --> It is better to give concentrations (mg/mL) per in2. Also, what is the solvent that authors used in drop casting method, they didn't mention the solvent even in the Materials and Methods section.

\blue{Once the both coatings cure, there is no liquid remaining on the fabric; therefore, mg/mL per in2 is not a valid unit of measurement of the coating on the fabric. The solvent is a proprietary blend for reactive silver ink while the solvent is sodium citrate for silver nanoparticles which is mentioned in the text. }

*Additionally, we investigate the anti-viral activity.... in solution (Fig 2c and S1 Fig) --> Fig 2c is believed to be Fig 2d. Moreover, S1 Fig gives information about Adenovirus type 5 which is not mentioned anywhere in the manuscript. Did authors give antiviral activity of silver sources against HAdV5 on purpose? How did\ they comment about the silver np acting better on enveloped virus then non enveloped one?

\blue{The adenovirus results have been removed for clarity. }

*Virus assay in saline are conducted by mixing silver ink or silver nanoparticles (0.1\%) in PBS...--> Is 0.1\% (v/v), (w/w) or (w/v)? it is better if authors specify the type of concentration.

\blue{0.1\% (v/v) is now specified in the Materials and Methods section. }

*Both PET-NP and PET-Ink samples are dipped in PDMS (1:10 ratio, Sylgard) and cured in oven ... --> What does 1:10 ratio stand for? Is it the PDMS: Hexane ratio or is it PET-NP or PET-Ink: PDMS ratio? This part is not clear. Authors should clarify this part. Fig 3a gives the schematic representation of PDMS treatment for silver coated PET samples and as it is understood from the figure PDMS is dissolved in Hexane however, authors did not mention about hexane in the article. The amounts and preparation steps of PDMS solution should be clarified. Also, Sylgard has various silicone elastomers with different codes. It is better to give the SylgardTM and code.

\blue{Sylgard 184 and a 1:10 ratio of curing agent:PDMS dissolved in hexane has been used. The description of PDMS has been modified within the text for clarity. }

*It would be great if the unwashed and washed fabrics antiviral activity tests were done simultaneously to keep the initial viral concentration same for all samples to see the difference between wash cycles. According to the data authors reveal for sample PET-NP/P antiviral activity is increased even a little with washing which seems not possible.

\blue{This is not statistically true, even though the overall mean may be slightly less. The groups are not statistically different.}

* For XRD peaks are observed between 20-50. Ag has two more major peaks at 64.7 and 77.4. It would be better to scan till 80$\degree$.

\blue{The authors choose to show the largest peaks in the 20-50 range. }

Materials and Methods

* To evaluate the antiviral activity of textile materials, the authors should follow"ISO 18184 – Determination of Antiviral Activity of Textile Products". How do the authors compare their results with ISO standards?

\blue{To our knowledge our method is nearly identical to ISO 1814. We used a 1 hour incubation period instead of 2 hours.}

Finally, it would be great if the authors could cite and discuss the following publication in their manuscript. https://link.springer.com/article/10.1007/s12010-016-2275-5
DOI 10.1007/s12010-016-2275-5

\blue{This work has been cited. }

\newpage
\textbf{Reviewer: 5}

\textbf{Recommendation:}
In the submitted manuscript, the authors showed that the silver amine complex ink, which contains reactive ionic silver, can coat PET fibres to create a low surface energy surface, superhydrophobic surface. The applied coating surface exhibits antiviral properties and is resistant against extended ultrasonic bleach washing. In the introduction section, the importance of this study has been discussed very well. The manuscript can be improved and become ready to publish if the authors can address these issues:

\textbf{Comments:} 

53 In this study the silver amine complex is being studied, however, this ink needs to be fully chemically characterised and formulated.

\blue{The silver amine complex has been fully chemically characterised and formulated in the cited paper$^{39}$. Additional formulation is held under intellectual property.}

98 “In contrast, PET fabrics coated with silver ink (PET-Ink) show more uniform coverage around the microfibers of the fabric (Fig 1d).”
The SEM images Fig 1c and Fig1d show that some parts of the fabrics are coated with either silver nanoparticles or silver ink, and the rest of the fabrics are intact.

\blue{We do not see issue here. }

101 The x-axis of the XRD pattern should change to (2$\theta$ (°)). Please at Y-axis please remove numbers if it is going to be reported by Arbitrary unit or remove (a.u.).

\blue{The axes have been changed. }

104 Author mentioned that the untreated PET is fully wetted by water and Fig 2 (a) shows that the water static, contact angle, for PET is zero. Although other studies such as the Indian Journal of Fibre \& Textile Research Vol. 37, September 2012, pp. 287-291 showed the water contact angle of untreated PET fabric is 124.24$\degree$.

\blue{The paper mentioned uses weaved PET fabric which may be more repellent. The PET used is knit cleanroom wipes, which are fabricated to have high sorbency. }

144. Could the authors please explain based on what parameters/reasons curing temperature sets at 150°C for 2h for a thin layer of PDMS?

\blue{The curing temperature was chosen to fully cure the PDMS. Lower temperatures did not fully cure the PDMS. }

263 What is the solid content of the silver ink?
The authors mentioned adding 0.2 mg of silver NP and silver ink (which contains amine groups (line 53). Therefore, how authors are certain that they added an equal amount of silver for silver NP and Silver ink samples?

\blue{The silver ink is 20\% silver in w/v which has been added to the text while the silver nanoparticles are 0.1\%. 1 \micro L of reactive silver ink or 200 \micro L of the silver nanoparticle solution was added. To the best of our knowledge, an equal amount of silver by weight is added. }

114. Since the content of silver in the PET ink and PET-NP are not equal, the conclusion from Fig 2b is not accurate.

\blue{To the best of our knowledge, an equal amount of silver by weight is added.}

261 What is the weight ratio of the silver ink to the PET fabric?

\blue{The weight ratio was not calculated. }

276 Please change wavelength to wavenumber.

\blue{This change has been made. }

\newpage


\textbf{Submitting Authors:}\\
%Sajad Haghanifar\\
%Benedum Hall, Room SB10 \\
%Pittsburgh, PA 15261\\
%Email:  \href{mailto:sah175@pitt.edu}{sah175@pitt.edu}\\
%Phone: (412) 499-0768
Anthony J. Galante\\
Benedum Hall, Room SB10 \\
Pittsburgh, PA 15261\\
Email:  \href{mailto:ajg109@pitt.edu}{ajg109@pitt.edu}\\
Phone: (919) 757-4183\\

Brady C. Pilsbury \\
University of Pittsburgh \\
Department of Industrial Engineering \\
Benedum Hall, Room SB10 \\
Pittsburgh, PA 15261\\

Kathleen A. Yates\\
 203 Lothrop Street, Floor 10\\
Pittsburgh, PA 15213, USA\\
Email:  \href{mailto:kayates@pitt.edu}{kayates@pitt.edu}\\

Melbs LeMieux \\
Electroninks Inc \\ 
7901 East Riverside Drive, Bldg 1,Unit 150\\
Austin TX  78744 \\

Eric G. Romanowski\\
%\altaffiliation{A shared footnote}
 203 Lothrop Street, Floor 10\\
Pittsburgh, PA 15213, USA\\
Email:  \href{mailto:egr1@pitt.edu}{egr1@pitt.edu}\\
%\phone{+123 (0)123 4445556}
%\fax{+123 (0)123 4445557}

Robert M. Q. Shanks\\
 203 Lothrop Street, Floor 10\\
Pittsburgh, PA 15213, USA\\
Email:  \href{mailto:shanksrm@pitt.edu}{shanksrm@pitt.edu}\\

Paul W. Leu\\
Benedum Hall, Room 1035 \\
Pittsburgh, PA 15261\\
Email:  \href{mailto:pleu@pitt.edu}{pleu@pitt.edu}\\
Phone: (412) 624-9834

\textbf{Corresponding Author:}\\
Paul W. Leu\\
Benedum Hall Room 1035 \\
Pittsburgh, PA, 15261\\
Email:  \href{mailto:pleu@pitt.edu}{pleu@pitt.edu}\\
Phone: (412) 624-9834





%\bibliographystyle{unsrt}
%\bibliography{AllRefs,LAMPPapers}


\end{document} 
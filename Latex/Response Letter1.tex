% request for removal from telemarketing lists

\documentclass[12pt]{letter}

\newcommand{\cX}{{\mathcal X}}

\usepackage{gensymb}
\usepackage[version=3]{mhchem} % Formula subscripts using \ce{}
%\usepackage{hyperref}
\usepackage[normalem]{ulem}
\usepackage{textcomp}
\usepackage{color}
\usepackage{xcolor}
%\usepackage{pdfcomment}
\usepackage{amsmath}
\usepackage{amssymb}
\usepackage{pdfcomment}

\usepackage{paralist}

\usepackage{setspace}
\usepackage{hyperref}
%\usepackage{color}
\usepackage{textcomp}
\usepackage{amsmath}
\usepackage{ulem}
\usepackage{graphicx}


%\definecolor{blue}{rgb}{0,0,0.543}
%\definecolor{blue}{rgb}{0,0.08,0.45}
%\def\blue[rgb]{0 0 139}

\oddsidemargin=.2in
\evensidemargin=.2in
\textwidth=5.9in
\topmargin=-.5in
\textheight=9in

\usepackage{xargs}
\usepackage[colorinlistoftodos,prependcaption,textsize=tiny]{todonotes}


%\newcommand{\blue}[1]{\textcolor{blue}{#1}} %for displaying red texts
%\newcommand{\strikeout}[1]{\textcolor{blue}{\sout{#1}}} %for displaying red texts


\newcommandx{\commentPaul}[2][1=]{\todo[linecolor=red,backgroundcolor=red!25,bordercolor=red,#1]{#2}}

\defineavatar{Paul}{color=red}


%\newcommandx{\commentPaul}[2][1=]{\todo[linecolor=red,backgroundcolor=red!25,bordercolor=red,#1]{#2}}


\newcommand{\strikeout}[1]{\textcolor{blue}{\sout{#1}}} %for displaying red texts

\newcommand{\red}[1]{\textcolor{red}{#1}} %for displaying red texts
\newcommand{\blue}[1]{\textcolor{blue}{#1}} %for displaying red texts
\newcommand{\Hmax}{\hat{H}}

\onehalfspacing
%\name{Sajad Haghanifar}
\name{Paul W Leu}

\date{\today}

\begin{document}

\begin{letter}{}

\opening{Dear Dr.~Ned Bowden:}


We thank the three reviewers for their evaluation, generally positive comments and conclusion that our work is of interest 
to the readership of ACS Applied Nano Materials.

% We thank the three reviewers for their evaluation of our manuscript and comments.  
%We thank the reviewers' statements that ``this review article is comprehensive to introduce the various sources of bio-inspiration, the critical performance requirements and the strategies that have been carried out to tackle the challenges" and we have provided ``an elaborate and comprehensive arrangement of various transparent substrates and barrier layers for optoelectronics." 

We have made improvements to our paper based on the reviewers' comments and suggestions.
Below, we summarize the changes that have been made and respond to the reviewers' comments.
Considering the importance of the topic of this
article on 
%for application of 
durable, anti-viral fabric coatings,
we resubmit this manuscript for your consideration of publication in ACS Applied Nano Materials.
Thank you for your consideration.

\closing{My best regards,}

%Paul Leu



\end{letter}



\newpage

\textbf{Editor}


\begin{enumerate}

\item 
Please add a 
scheme near the 
beginning (probably 
defined as Scheme 1) 
to illustrate the 
system 
(structure/chemical 
composition) and the 
synthesis/fabrication 
from a chemistry 
perspective.

\blue{A schematic showing the synthesis and structure of the functionalized nano graphene oxide has been added to Figure 1.}

\item 
Please remove the box 
around the TOC graphic and 
increase its clarity/resolution.

\blue{The box has been removed and the resolution of the graphic has been improved by using a .tif file.}

\item In Figures 1c, 5c, S5, and 2c please use a.u. for the units and not use A.u.

\blue{This has been corrected.}

\item The contact angle data in Figures 3a and 5e are reported with too much precision (i.e., too many significant figures). Contact angles should be reported as xx degrees not xx.y degrees. Please check carefully and revise where needed. 

\blue{This has been corrected.}

\item Please check Figure S4, some of it is missing.

\blue{This has been corrected.}

\item Please reformat the Supporting Information so that each figure appears on the same page as its figure caption.

\blue{This has been corrected.}

\end{enumerate}

\newpage

\textbf{Reviewer: 1}

\textbf{Recommendation:} Publish after minor revisions noted.

This manuscript reports a new strategy for bleach washable and durable anti-virofouling fabiric coating using functionalized graphene oxide derived from coals. This strategy enables to easily endow an antiviral property with the fabric used here based on simple and cost-effective processes. The coated functionalized graphene oxides attached on the surface of the fabrics show good anti-virofouling property and durability. In my view, the results in this manuscript is publishable in this prestigious journal after major revision.

\blue{We thank the reviewer for his or her positive comments and evaluation of our article to be publishable in ACS Applied Nano Materials.
We have made improvements based on the reviewer's comments as shown below.}
%\commentPaul{Think all page numbers and paragraph numbers need to be checked and changed as I think they have changed after modifying figures.}

% to introduce the various sources of bio-inspiration,the critical performance requirements and the strategies that have been carried out to tackle the challenges" and the content we provide is “solid and well drafted". We have made improvement based on reviewer's comments:}


\begin{enumerate}
\item The coal-derived graphene oxide has $\sim$8 at.\% of oxygen content. Actually, this amount of oxygen content is not sufficient to call this material graphene oxide. I am not raising a big issue on the this “graphene oxide”. But, if the authors can search a more proper name for their materials, it can be changed.

\blue{The low oxygen content is due to the functionalization of the graphene oxide (GO). The graphene oxide oxygen content before functionalization is about 34.5\%, but after functionalization much of the oxygen content is lost due to the reaction to form amide between carboxylic groups on the GO and the amines.
%and the loss of 
% there is loss of O (formation of water between COOH and NH$_2$), etc., and so you’ll see a decreased oxygen content in the functionalized GO. 
This explanation has been added to the end of page 7, line 15-20. 
%\commentPaul{Add verbatim the text that has been added so the reviewers don't have to go searching for it.}
Additionally, the text has been corrected throughout the paper to refer to the nanomaterial as functionalized nano-graphene oxide for clarity.}
%\commentPaul{In some places, you use nano graphene oxide.  In others nano-graphene oxide.  It needs to bee consistent.  Please check that ``nano-graphene oxide" is used throughout.}
%\commentPaul{Perhaps, we can just call it graphene oxide?  I'm not sure why we want the prefix nano-?}


\item The mechanism for antiviral property of the GO-ODA need to be explained. Does the anti-fouling or antiviral properties of the materials originate from oxygen function group or ODA functional group?

\blue{ The properties come from the ODA functional group and nGO nanostructure. The discussion on page 20, paragraphs 1-3 have been modified for clarity. 
%\commentPaul{Add verbatim the last to paragraphs so the reviewers don't have to go searching for it.}
The anti-viral behavior observed 
%is anti-virofouling which 
comes from the repellency and inactivation property of the ODA functional group combined with the nanoroughness from the nGO.}
%The virus may not have been killed, but you we can accurately state that infectivity of the virus was reduced. This is just as good as killing the virus. 
%When we dilute the virus out from the original samples, there is no increase in virus in higher dilutions compared to decreased virus in the lower dilutions suggesting a virus inhibition of infectivity effect. }


\item How about the anti-fouling or antiviral property of GO-ODA compared with conventional GOs? Please explain the advantages of GO-ODA compared to conventional GO or nitrogen doped GO.

\blue{The following text was added to page 4, paragraph 2 ``nGO-ODA offers the advantage of simultaneous liquid repellency and viral inhibition properties compared to conventional GO or nitrogen doped GO, which only demonstrate virus inhibition."}


\end{enumerate}

\newpage
\textbf{Reviewer: 2}

\textbf{Recommendation:} Major revisions needed as noted.

\blue{We thank the reviewer for his or her constructive comments. % and evaluation of our article to be published in this prestigious journal.
We have made improvements based on the reviewer's comments as shown below.}


\begin{enumerate}

\item    Page 9 line 8 “This provides ………. wetting property.” It seems like roughness is provided for the dip coating of nGO-ODA inspite of superhydrophobic wetting property. Please confirm as optical image in fig 3c ii contradicts this statement.

\blue{The paragraph on page 8-9 paragraph 1 has been modified for clarity.}
%\commentPaul{Add verbatim the paragraph so the reviewers don't have to go searching for it.}


\item   What is the method of curing to make etched PET nGO-ODA?

\blue{The etched PET nGO-ODA was cured by oven curing.  %Oven curing has been added to the text. 
The details of oven curing are provided in the experimental section under sample preparation.}  %\commentPaul{Are details provided on the conditions for oven curing?  Please include here.}

\item     In fig 2 b the scales are not visible

\blue{Scale bars have been added.}


\item     Please elaborate how the sem images proves/confirms the etching is done because the schematic of EPET 2aii shows some structure is growing.

\blue{The schematic is meant to show additional roughness on the microfibers by etching, not growth. The SEM image Figure 2bii inset shows this additional roughness on the PET microfiber. The text on page 8-9, paragraph 1 has been modified for clarity. }
%\commentPaul{Provide the new text so the reviewers don't have to go searching for it.}

\item    Line 33 page 9 “The terephthalate group is confirmed at wavenumber cm-1.” Is some value missing here?

\blue{The value 1240 cm-1 has been added. }

\item     The FTIR  of Fig. 2c are on different scales so the stretching and other details could not be confirmed. In addition what does the lowering/enhancing of surface energy signifies with methyl groups or any other group other than chemical resistance?

\blue{The text has been modified on page 17, paragraph 1, ``Additional methyl groups from the PDMS treatment are observed at wavenumber 2850 cm$^{-1}$. Methyl groups lower the surface energy of the fabric which improves the liquid repellency property."}

\item    Fig 3a looks more of a table rather than an image. It could be marked as table instead

\blue{The authors intend for the table to be a part of Figure 3. }

\item     The different angles named in the fig 3a is not discussed much. It could be explained for a better understanding of the importance of the angles for the study of fabric behaviour.

\blue{ The following text has been added to page 10, paragraph 1, ``The static contact angle is the angle at which the contact area between the liquid and solid is unmoved. The advancing angle describes the angle when a droplet is wetting the surface and the receding angle is the angle when a droplet is dewetting the surface. The hysteresis describes how easily a droplet rolls off the surface and is calculated by the difference between the advancing and receding contact angles."}

\item    what kind of transition is Cassie-Baxter to Wenzel wetting state and how effective it is and the cause behind this transition and its role in demonstrating anti virofouling properties?

\blue{The following text has been added on page 11, paragraph 1, 
``Untreated PET and etched PET samples are fully wetting with water droplets; faster liquid spreading was observed for etched PET fibers.
After treatment, the 
EPET-nGO samples are superhydrophobic, with an average water static contact angle of $156 \pm 1\degree$ and water hysteresis of $9 \pm 1\degree$.
The combination of roughened microfibers and nGO-ODA nanomaterial creates a low surface energy surface with dual micro- and nano-scale roughness, important for rendering a robust, superhydrophobic wetting property.$^{30,31}$
The Wenzel wetting state is when the liquid is in fully wetting and in full contact with the roughness of a surface. 
After etching and nGO-ODA coating, a low energy surface with both micro- and nano-scale roughness is created that promotes Cassie Baxter wetting instead of Wenzel wetting.  
The Cassie-Baxter wetting state is when liquid sits on top of the roughness of a surface, resulting in air pockets between the surface and the liquid. 
The low contact fraction area between the surface and liquid results in high static contact angle as well as low contact angle hysteresis characteristic of superhydrophobic surfaces, where droplets ball up on the surface and easily roll off.  
The amount of Laplace pressure required for liquid to infiltrate the air pockets and transition from Cassie-Baxter  to Wenzel wetting state is called the breakthrough pressure. The breakthrough pressure is a measure of the stability of the Cassie-Baxter wetting state."}



%``The Wenzel wetting state is when the liquid is in full contact with the roughness of a surface. The Cassie-Baxter wetting state is when liquid sits on top of the roughness of a surface, resulting in air pockets between the surface and the liquid. 
%The amount of Laplace pressure required for liquid to infiltrate the air pockets and transition from Cassie-Baxter  to Wenzel wetting state is called the breakthrough pressure. The breakthrough pressure is a measure of the stability of the Cassie-Baxter wetting state."}

%``The Wenzel wetting state is when liquid is in full contact with the roughness of a surface. The Cassie-Baxter wetting state is when liquid sits on top of the roughness of a surface, resulting in air pockets between the surface and the liquid. 
%The amount of Laplace pressure required for liquid to infiltrate the air pockets and transition from Cassie-Baxter  to Wenzel wetting state is called the breakthrough pressure. Breakthrough pressure is a measure of the stability of the Cassie-Baxter wetting state and we believe anti-virofouling properties occur when a surface maintains a robust Cassie-Baxter wetting state." }

\item   Fig 3 c) Optical images of water (blue) and human saliva (yellow) droplets on i) i) untreated PET

\blue{The unnecessary i) has been removed. }

\item   What is the difference and significance of Taber rotary abrasion cycles and bleach washing durability?

\blue{The statement ``Taber rotary abrasion is a standard method for analyzing the durability of fabric coatings under mechanical wear and bleach washing is a standard protocol for re-usable, personal protective equipment in healthcare settings." has been added to Page 11 paragraph 3.} 

\item   Page 11 line 18 “Fig .3d ………… and ultrasonic bleach washing …………… Fig .3e …….a function of Taber rotary abrasion cycles.” Revisit line 34 for the same

\blue{These typos have been corrected. }

\item  Page 11 line 39 “to dip coating …………….improves the stability of nGO-ODA on” what kind of stability is being discussed, mechanical thermal chemical?

\blue{``Wetting stability" has been added to the text. }

\item  Page 11 line 41 “Fig. 4b shows representative optical images of water” please verify.

\blue{This typo was replaced with Fig. 3e. }

\item   What is PBS and please check the namings of figure 4

\blue{``phosphate buffered saline (PBS)" has been added. }

\item   It is not clearly discussed about the EPGO or EPGOP treatment and what does the decrease in $p <0:05$ to $p <0:001$, signifies ?

\blue{The following was added to Page 13, paragraph 1,
``Mann-Whitney tests were used to assess the probability that that the fabric treatment reduces the amount of virus particles on the fabric compared to controls. 
%offer a degree of statistical certainty that the fabric treatment reduces the amount of virus particles on the fabric compared to controls. 
%Mann-Whitney tests offer a degree of statistical certainty that the fabric treatment reduces the amount of virus particles on the fabric compared to controls. 
Asterisks indicate the 
%probability with 
level of statistical significance with 
which the treatment
group is 
%the same as the 
different 
from the control group %via Mann-Whitney statistic tests 
with one asterisk for $p \leq 0.05$, two for $p < 0.01$ and three for $p < 0.001$.} %One asterisk corresponds to the level of certainty $p \leq 0.05$, two asterisks for $p < 0.01$ and three asterisks for $p < 0.001$. 
%Asterisks indicate the level of statistical significance with which the treatment
%group different from the control group via Mann-Whitney statistic tests with one asterisk for $p \leq 0.05$, two for $p < 0.01$ and three for $p < 0.001$.} %One asterisk corresponds to the level of certainty $p \leq 0.05$, two asterisks for $p < 0.01$ and three asterisks for $p < 0.001$. }

%``Mann-Whitney tests offer a degree of statistical certainty that the fabric treatment reduces the amount of virus particles on the fabric compared to controls. Asterisks indicate the statistical certainty that the control group is the same as the treatment from Mann-Whitney statistic tests, one asterisk for $p \leq 0.05$, two for $p < 0.01$ and three for $p < 0.001$." The EPGO or EPGOP was a typo that has been corrected. The decrease in p value is just a slight difference in statistical certainty.  }
%\commentPaul{statistical significance, right?}

\item   How 10\% bleach solution was chosen for bleach washing durability tests?

\blue{The following was added to the experimental section under Durability testing ``A concentration of 10\% bleach washing solution was chosen based on discussions with healthcare cleaning provider Cintas."}

\item   What is the requirement for PDMS to dilute in hexane?

\blue{The following was added to the experimental section for sample fabrication, ``Hexane was chosen as a solvent due to its low surface tension to easily wet the entire fabric surface during dip coating."}

\item       Is there any trial-and-error experiment were conducted for this study?

\blue{ This study focuses on the bleach durability of the fabric coating. There are trial and error experiments in this regard. The fabric with just nGO-ODA did not show bleach durable liquid repellent properties. The fabric was then etched prior to improve the liquid repellency durability of the coating; however, even with the additional etching step, significant color fastness was observed, shown in Figure 3e. Finally, a PDMS layer was added to improve the bleach wash durability by adding chemical resistance. }

\item       Errors in the figure name, and are not arranged properly according to the content in result and discussion part, it makes difficult to read.
with respect to electrical, thermal, or mechanical stability?

\blue{Figure names have been changed accordingly.}

\item  Graph scale is not uniform, difficult to identify the variations in the bar graphs.

\blue{The graph scale is uniform for each type of virus. The titers were different for each virus as each type of virus has a different level of viability and infectivity. }

\end{enumerate}

\newpage
\textbf{Reviewer: 3}

\textbf{Recommendation:} Major revisions needed as noted.



In the manuscript entitled “Coal-Derived Functionalized Graphene Oxide for Bleach Washable, Durable Anti-Virofouling Fabric Coatings”, the author reported the personal protective equipment material with the antiviral resistant. The author claiming that octadecylamine functionalized graphene-oxide/PDMS based PET fabric shows the good repellency and durability with water, human saliva and viruses. However certain aspects the authors have to modify the manuscript to improve the scientific quality. 

\blue{We thank the reviewer for his or her constructive comments.
We have made improvements based on the reviewer's comments as shown below.}

\begin{enumerate}
\item The author didn’t explain properly the role of each material in the composite fabrics. The author have to elaborately explain the role of the materials like octadecylamine functionalization with GO, PET-nGO-ODA fabric and PET nGO-ODA/PDMS composite fabric.

\blue{
The text has been modified for 
clarity. On page 4, paragraph 2 the role of octadecylamine functionalized nano graphene oxide (nGO-ODA) is explained ``In this work, we demonstrate a nano-graphene oxide (nGO) that is coal-derived, 
as opposed to graphite-derived and is 
thus, cheaper and more realizable for large scale applications such as fabrics.The nGO is functionalized with long hydrocarbon chains from octadecylamine (ODA) to make  octadecylamine functionalized nano-graphene oxide (nGO-ODA). The ODA functional group adds a liquid repellent property to the nGO from low surface energy, hydrocarbon chains. nGO-ODA offers the advantage of simultaneous liquid repellency and viral inhibition properties compared to conventional GO or nitrogen doped GO, which only demonstrate virus inhibition."}




\blue{The role of PET, etched PET and etched PET-nGO-ODA is explained on page 8 paragraph 1 ``In this study, we use polyester (PET) as fabric material since it is commonly used for medical and healthcare applications such as gowns, scrubs and caps...Roughening the microfibers prior to coating renders the surface superhydrophobic once the nGO-ODA coating is added...The additon of nGO-ODA material lowers the overall surface energy and adds roughness at the nanoscale, important for repelling small scale (less than 600 nm) particulates.$^{29}$"}



\blue{The role of etched PET-nGO-ODA/PDMS is explained on page 16 paragraph 1``Since some of the nGO-ODA material on EPET-nGO samples can be removed from the fabric when subject to extended washing cycles with bleach, a thin layer of PDMS is applied to the fabric afterwards to improve the bleach wash durability by adding chemical resistance. The PDMS adds chemical resistance and durability so the treatment may be re-usable in bleach washing procedures."}


\item     In the preparation process 200mg of octadecylamine used in this work and in the XPS spectrum elemental analysis (Figure 1d) are also shows the minimum amount of addition of the amine group in the functionalization. At what measure the author fixed the particular concentration and justify with the experiment with concentration variation with the functionalization and concentration variation of the composite.

\blue{The ODA:nGO ratio for functionalization was chosen to be 2:1.  The following text has been added to the experimental information under Preparation of nGO-ODA ``A 2:1 ratio of ODA:nGO was observed to be the ideal ratio for a coating solution. At lower ratios, the solution aggregates in organic solvents and the graphene flakes are not fully functionalized. At larger ratios, the solution is mostly amines and too viscous for coating. The ratio was chosen based on observational experiments in the lab." }
%\commentPaul{Think more information in the text similar to the explanation here.}

\item     The authors reported the SEM image of the PET, EPET and EPET-nGO sample. Why the scale is deleted in the image? The author should report with the correct scale.

\blue{The scale was omitted by accident and now has been added.}

\item      In the figure 2C the FTIR spectrum of the different samples reported. But amine group vibration is not identified in the sample. The author have to justify the same with the clear explanation.

\blue{The following text was added to the end of page 9, last sentence, ``The amine group vibration from the GO-ODA can not be identified due to the dominant signal from the hydrocarbon groups of the polyester substrate."}

\item      The virus repellant property of the fabric have been tested and reported in the figure 4 and 6. What is the real property associated with the virus repulsion other than the phobicity. The author have to explain clearly.

\blue{Page 18 paragraph 2 has been modified ``The reduction of virus quantities from anti-virofouling comes from the stable Cassie-Baxter wetting state. In a stable Cassie-Baxter wetting state, the area of fabric contact for virus particles in liquid to contaminate the fabric surface is significantly less due to the trapped pockets of air.$^{11}$ The  combination of nanoroughness from the 2 dimensional graphene lattices and low surface energy from the ODA functional groups promote self-cleaning of extremely small (less than 600 nm) contaminants, such as viruses by reducing the area of contact.$^{29}$  The treated surface prevents liquid penetration and significantly reduces virus-to-surface contact area; therefore, less virus particles may attach to the surface. "}

\item    In the figure 6, the virus quantity have been reduced, which suggesting that the fabric have the virus resistance. Usually, viral resistance can be caused with the protein leakage or structural breakage. The author have to identify the mechanism of the viral resistance with essential proof.\\
\blue{This study focuses on the durabilitiy of the virus functionality on fabric. In our experiments, we observed that the textile coatings are antiviral from a combination of reducing viral adhesion to the surface and  
inhibiting virus infectivity.  
%virus inhibition of infectivity effect, where the infectivity of the virus is inhibited. This is just as useful as destroying the virus; however, the mechanism of the inhibition is beyond the scope of this work 
The reduction of virus infectivity was observed with the nGO-ODA material in PBS and on fabrics. 
%for HAdV5. to the surface appears to be dominant though as virus inhibition experiments demonstrated a reduction of HAdV5 by $0.88 \pm 0.64$ log$_{10}$
%compared to $1.8$ log$_{10}$
%with EPET-nGO/P samples.  
However, the mechanism of the inhibition is beyond the scope of this work and can be identified in future studies.  The literature has mentioned both mechanical damage and/or oxidative stress as antiviral mechanisms, but there is currently no consensus.} 
%\commentPaul{Why?  Explain what this would entail and why it is beyond the scope of this work.}
%and may be explained in future studies.} 
%\commentPaul{Not sure how we concluded this.  0.88 log is much smaller than 1.8 log?}
% The dominant property is anti-virofouling, though there is also some inhibition.}


%\blue{``Virus resistance" should be used in this case. Viral resistance to an agent is acquired over a period of time after exposure to the agent in which there is a decreased susceptibility over time. These viruses were never exposed to the agents prior to the assay and therefore resistance does not occur. We did observe a virus inhibition of infectivity effect, where the infectivity of the virus is inhibited. This is just as useful as destroying the virus; however, the mechanism of the inhibition is beyond the scope of this work and may be explained in future studies. The dominant property is anti-virofouling, though there is also some inhibition.}
%This effect is small and we conclude the dominant property occurring is anti-virofouling. }




\end{enumerate}



\blue{In addition to these changes, we have also made minor changes throughout the paper in order to improve its quality.}



\newpage


\textbf{Submitting Authors:}\\
%Sajad Haghanifar\\
%Benedum Hall, Room SB10 \\
%Pittsburgh, PA 15261\\
%Email:  \href{mailto:sah175@pitt.edu}{sah175@pitt.edu}\\
%Phone: (412) 499-0768
Anthony J. Galante\\
Benedum Hall, Room SB10 \\
Pittsburgh, PA 15261\\
Email:  \href{mailto:ajg109@pitt.edu}{ajg109@pitt.edu}\\
Phone: (919) 757-4183\\

Brady C. Pilsbury \\
University of Pittsburgh \\
Department of Industrial Engineering \\
Benedum Hall, Room SB10 \\
Pittsburgh, PA 15261\\

Kathleen A. Yates\\
 203 Lothrop Street, Floor 10\\
Pittsburgh, PA 15213, USA\\
Email:  \href{mailto:kayates@pitt.edu}{kayates@pitt.edu}\\

Melbs LeMieux \\
Electroninks Inc \\ 
7901 East Riverside Drive, Bldg 1,Unit 150\\
Austin TX  78744 \\

Eric G. Romanowski\\
%\altaffiliation{A shared footnote}
 203 Lothrop Street, Floor 10\\
Pittsburgh, PA 15213, USA\\
Email:  \href{mailto:egr1@pitt.edu}{egr1@pitt.edu}\\
%\phone{+123 (0)123 4445556}
%\fax{+123 (0)123 4445557}

Robert M. Q. Shanks\\
 203 Lothrop Street, Floor 10\\
Pittsburgh, PA 15213, USA\\
Email:  \href{mailto:shanksrm@pitt.edu}{shanksrm@pitt.edu}\\

Paul W. Leu\\
Benedum Hall, Room 1035 \\
Pittsburgh, PA 15261\\
Email:  \href{mailto:pleu@pitt.edu}{pleu@pitt.edu}\\
Phone: (412) 624-9834

\textbf{Corresponding Author:}\\
Paul W. Leu\\
Benedum Hall Room 1035 \\
Pittsburgh, PA, 15261\\
Email:  \href{mailto:pleu@pitt.edu}{pleu@pitt.edu}\\
Phone: (412) 624-9834





%\bibliographystyle{unsrt}
%\bibliography{AllRefs,LAMPPapers}


\end{document} 